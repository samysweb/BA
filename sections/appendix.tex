%\chapter{Appendix}
%\label{chap:appendix}

\chapter{Reproducibility}
\label{sec:appendix:reproducibility}
\paragraph{Software}
For alle experiments a modified version of \texttt{Boolector 3.0.1-pre} is used.
More specifically we modified commit \texttt{f689fbbfe820392d35e26be368f9d87d2dbdb037} so that we could measure the time of the \texttt{check-sat} instruction.
As underlying SAT-solver Lingeling \cite{Biere-SAT-Competition-2017-solvers} with version \texttt{bcj 78ebb8672540bde0a335aea946bbf32515157d5a} is used.
All software packages were compiled using the provided cmake scripts which have the highest optimization levels enabled using gcc in version \texttt{(Ubuntu 5.4.0-6ubuntu1~16.04.10) 5.4.0 20160609}.\\
For the final experiments presented in Chapter \ref{ch:evaluation} Ablector is used in the version available in commit 79584caeb4b7ea27ac3e80153b167c36d434232e at \url{https://github.com/samysweb/ablector}.

\paragraph{Machine}
All experiments were executed on a cluster of 20 identical compute nodes each housing 2 Intel Xeon E5430 @ 2.66GHz CPUs and a total of 32GB of RAM.
The SMT benchmark files were stored on a RAID system connected to the cluster.

\paragraph{Benchmark execution}
2 jobs were run in parallel on each compute node with the timeout set to 1200 seconds
this posed no caching issues as they were run on seperate CPU sockets
\footnote{Early on we ran up to 8 experiments on a single node to make use of the available cores however this seemed to produce caching issues slowing down the experiment times}.
For time surveillance and measurements we used the runlim utility \cite{runlim}.
All benchmarking scripts and the log results can be obtained at \todo{Insert experiment repo once uploaded}.
\todo{Publish repos!}

\renewcommand{\simplechapterdelim}{.}
\listoffigures