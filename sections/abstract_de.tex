%% LaTeX2e class for student theses
%% sections/abstract_de.tex
%% 
%% Karlsruhe Institute of Technology
%% Institute for Program Structures and Data Organization
%% Chair for Software Design and Quality (SDQ)
%%
%% Dr.-Ing. Erik Burger
%% burger@kit.edu
%%
%% Version 1.3.3, 2018-04-17

\Abstract
Entscheidungsverfahren für SMT-Probleme in der Bitvektor-Theorie sind ein wesentlicher Bestandteil moderner Soft- und Hardware-Verifizierer.
In dieser Arbeit untersuchen wir, ob Abstraktionsverfahren bei der Lösung von Bitvektor SMT-Problemen helfen können.
Nach einem kurzen Überblick über aktuelle Abstraktionstechniken stellen wir einen neuartigen Lösungsansatz für die quantorenfreie Bitvektortheorie (\texttt{QF\_BV} in SMT-LIB) vor, der auf inkrementellem SMT solving und Abstraction Refinement basiert.
Wir implementieren diesen Ansatz in einem Prototyp zur Erweiterung des SMT-Solver Boolector und evaluieren seine Leistung anhand der relevanten Benchmark-Teilmenge der SMT COMP 2018.
Im Vergleich zu Boolector zeigt der neue Ansatz eine bessere Leistung bei unerfüllbaren Benchmarkinstanzen, während er bei erfüllbaren Instanzen schlechter ist.
Schließlich schlagen wir verschiedene Methoden vor, um die Leistung, insbesondere für erfüllbare Fälle, zukünftig zu verbessern.