\chapter{Preliminaries}
\label{ch:preliminaries}
In this chapter we will introduce the foundations of Boolean algebra and SAT solving. Based on these techniques we will introduce SMT solving and present two predominant SMT solving approaches.

\section{Basic boolean algebra and notation}
The structure and notation of this section is based on \cite{Marques-Silva-PropositionalSATSolving} and \cite{fondements-logique}.
\begin{definition}[Boolean variable, Atom, Literal]

\begin{itemize}
    \item A \textit{boolean variable} (represented by lowercase latin letters) is a variable which can be either \true or \false.
    \item An \textit{atom} is a boolean variable.
    \item A \textit{literal} $l$ is a boolean variable $x$ or its complement $\neg x$.
\end{itemize}

\end{definition}
\begin{definition}[Boolean formula]
\begin{itemize}
    \item An atom, $\square$ (false) and $\blacksquare$ (true)  are \textit{boolean formulae}.
    \item If $\mathcal{F}$ is a \textit{boolean formulae} so is $\left(\neg\mathcal{F}\right)$.
    \item If $\mathcal{F}$ and $\mathcal{G}$ are \textit{boolean formulae} so are $\left(\mathcal{F} \land \mathcal{G}\right)$ and $\left(\mathcal{F} \lor \mathcal{G}\right)$.
\end{itemize}
From hereon, we call the set of all Boolean formulae $\textsc{For}^0$.\\
\end{definition}
We define $var(\mathcal{F})$ as the set of all variables appearing in  $\mathcal{F}$.\\
For the rest of this chapter $\mathcal{F}$ and $\mathcal{G}$ will be assumed to be Boolean formulae.

\begin{definition}[Interpretations and Models]
An \textit{interpretation} $I$ for $\mathcal{F}$ is a mapping $I\colon var(\mathcal{F}) \to \{0,1\}$.
For some interpretation $I$ the value of $I(\mathcal{F})$ is defined inductively through:
\begin{itemize}
    \item $I\left(\square\right)=0$ and $I\left(\blacksquare\right)=1$.
    \item For $\neg\mathcal{G}$ the value is $\left(1-I(\mathcal{G})\right)$.
    \item For $\mathcal{G}_1 \land \mathcal{G}_2$ the value is $1$ iff $I(\mathcal{G}_1)=1$ and $I(\mathcal{G}_2)=1$ , otherwise $0$.
    \item For $\mathcal{G}_1 \lor \mathcal{G}_2$ the value is $1$ iff $I(\mathcal{G}_i)=1$ for either or both $i\in\{1,2\}$, otherwise $0$.
\end{itemize}

Some interpretation $I$ is a model for $\mathcal{F}$ iff $I(\mathcal{F})=1$ which we denote as $\mathcal{I}\vDash\mathcal{F}$.\\
A formula $\mathcal{F}$ is called satisfiable if there exists a model $\mathcal{I}$ for $\mathcal{F}$.
\end{definition}

Logic formulae with connectors like $\iff$, $\implies$ etc. can similarly be defined as Boolean formulae however this is not strictly necessary as any boolean function can be represented by a formula only using the connectors described above. Parentheses may be omitted in which case $\neg$ takes precedence over $\land$ takes precedence over $\lor$.

\begin{definition}[Conjunctions, Disjunctions and Clauses]
We call $\mathcal{F}\land\mathcal{G}$  a \textit{conjunction} and $\mathcal{F}\lor\mathcal{G}$ a \textit{disjunction}.\\
A \textit{clause} is a disjunction of literals.\\
A formula is in \textit{conjunctive normal form} (CNF) if the formula is a conjunction of clauses.\\
A formula is in \textit{disjunctive normal form} (DNF) if the formula is a disjunction of conjunctions.
\end{definition}
\begin{example}
$\left(a \lor \neg b \lor c\right)\land\left(d \lor \neg e\right)\land\left(g \lor b \lor c\right)$ is in CNF and\\
$\left(a \land \neg b \land c\right)\lor\left(d \land \neg e\right)\lor\left(g \land b \land c\right)$ is in DNF.\\
\end{example}



\section{Satisfiability and SAT solving}
The \textsc{NP}-complete problem of Satisfiability (\textsc{Sat}) concerns with whether a given Boolean formula $\mathcal{F}$ in CNF is satisfiable or not.
A decision procedure for the \textsc{Sat} problem is called a SAT solver.
For a proof on the \textsc{NP}-completeness of \textsc{Sat} the reader is referred to \cite{Garey-Intractability}.

\paragraph{DPLL} One of the first well-known algorithms for solving SAT problems is the Davis\hyp{}Putnam\hyp{}Logemann\hyp{}Loveland (DPLL) algorithm \cite{DPLL} of which many variants are still used today.
Generally speaking DPLL works by:
\begin{itemize}
    \item[A)]   \textbf{Unit Propagation of single literal clauses}\\
                Any literal which appears as the only literal in a clause may be assigned so that the clause evaluates to true
    \item[B)]   \textbf{Pure literal deletion}\\
                Any literal which only appears as positive (or negative) literal may be fixed
    \item[C)]   \textbf{Case splitting}\\
                If rules A) and B) can no longer be applied, some literal $l$ is chosen and the problem is split up in the case where $l$ is true and in the case where $l$ is false.
\end{itemize}

\paragraph{CDCL} Variants of the DPLL-algorithm have been extended by functionality to \enquote{learn} from contradictions encountered during search. This technique is called Conflict\hyp{}Driven\hyp{}Clause\hyp{}Learning (CDCL) and works by adding a clause which makes sure to avoid the encountered contradiction during further search. For details on the methodology the reader is referred to \cite{Grasp}.

\paragraph{Unsatisfiability Cores}
Another feature widely implemented in SAT solves is the ability to produce \textit{Cores} for unsatisfiable instances.
This allows the SAT solver to return the subset of formulae which was used to  produce the contradiction making the instance at hand unsatisfiable. This feature is used in a wide range of applications including model checking, debugging specification and abstraction refinement \cite{Marques-Silva-PropositionalSATSolving}.

\paragraph{}
Despite its theoretical hardness and the lack of known polynomial time algorithms for the \textsc{Sat} Problem, much progress has been made on practical \textit{SAT solving} using variants of the DPLL-algorithm and CDCL.
Among other applications those advances are particularly useful for the solving of so called SMT problems which are described in the next section.


\section{Satisfiability modulo theory}
The previous sections described various aspects of Boolean Algebra.
For many applications, especially in software verification, however, a problem description in a more powerful language is much more desirable. For example, describing the manipulation of datastructures such as bitvectors and arrays is better comprehensible and a lot more concise than describing the manipulation of isolated bits in memory.
This gives rise to the concept of \textit{Satisfiability Modulo Theory} (SMT) which restricts first-order logic or higher-order logics to syntactic or semantic fragments offering a good trade-off between the language's expressiveness and the ability to automatically check an instance's satisfiability.
Such fragments can then be decided by specialized decision-procedures exploiting properties of the specific sublanguage to enhance the solvers practical efficiency despite high worst-case computational complexity \cite{Barrett-Tinelli-SMT}. This section will give an overview over SMT and then describe 3 specific theories in more detail.

\subsection{The SMT language}
In a first step, this section introduces the many-sorted first-order logic as described in \cite{Barrett-Tinelli-SMT} with some inspiration from \cite{fondements-logique}.

\begin{definition}[Signature]
Given an infinite set $\mathbf{S}$ of \textit{sort symbols} and an infinite set $\mathbf{X}$ of \textit{variables} each assigned a sort $s\in\mathbf{S}$, a \textit{signature} $\Sigma$ consists of a tuple $\left(\Sigma^S, \Sigma^P, \Sigma^F, \mathcal{s}^P, \mathcal{s}^F \right)$ where:
\begin{itemize}
    \item $\Sigma^S \subseteq \mathbf{S}$ is a set of sort symbols;
    \item $\Sigma^P$ is a set of predicate symbols;
    \item $\Sigma^F$ is a set of function symbols;
    \item $\mathcal{s}^P\colon \Sigma^P \to \left(\Sigma^S\right)^\ast$ is a mapping from predicate to predicate sort; and
    \item $\mathcal{s}^F\colon \Sigma^F \to \left(\Sigma^S\right)^+$ is a mapping from function to function sort.
\end{itemize}
\end{definition}

\begin{definition}[Arity and Rank]
For a function $g\in\Sigma^F$ with $\mathcal{s}^F(g)=\sigma_1\dotsi\sigma_n\sigma$ the \textit{rank} of $g$ is defined as $\sigma_1\dotsi\sigma_n\sigma$. We call $n$ the \textit{arity} of $g$ (note that the arity may be $0$).\\
For a predicate $p\in\Sigma^P$ with $\mathcal{s}^P(p)=\sigma_1\dotsi\sigma_n$ the \textit{rank} of $p$ is defined as $\sigma_1\dotsi\sigma_n$ . We call $n$ the \textit{arity} of $p$ (as before the arity may be $0$ in which case $\mathcal{s}^P(p)$ is the empty word).
\end{definition}

\begin{definition}[Sorted $\Sigma$-terms]
A \textit{$\Sigma$-term} of sort $\sigma$ is either a variable $x\in\mathbf{X}$ of sort $\sigma$ or an expression $g\left(t_1,\dots,t_n\right)$ where $g\in\Sigma^F$ and $\mathcal{s}^F(f)=\sigma_1\dotsi\sigma_n\sigma$ with $\Sigma$-terms $t_i$ of sort $\sigma_i$ for every $i\in\llbracket1,n\rrbracket$.\\
For any $\sigma\in\Sigma^S$ we define $\textsc{Term}^\Sigma_\sigma$ as the set of all $\Sigma$-terms of sort $\sigma$.
\end{definition}

\begin{definition}[Atomic $\Sigma$-formulae, $\Sigma$-literals]
\textit{Atomic $\Sigma$-formulae} are:
\begin{itemize}
    \item The symbols $\square$ (false) and $\blacksquare$ (true);
    \item Expressions $t_1 \doteq t_2$ with $t_1, t_2$ $\Sigma$-terms of the same sort $\sigma\in\Sigma^S$; and
    \item Expressions $p\left(t_1,\dots,t_n\right)$ with $p\in\Sigma^P$, $\mathcal{s}^P(p)=\sigma_1\dotsi\sigma_n$ and $t_i$ of sort $\sigma_i$ for every $i\in\llbracket1,n\rrbracket$.
\end{itemize}
In correspondence with the previous definitions of literals in Boolean Algebra a $\Sigma$-literal is an atomic $\Sigma$-formula $\phi$ or its complement $\neg\phi$.
\end{definition}

\begin{definition}[$\Sigma$-formulae]
Analogue to Boolean Algebra the \textit{$\Sigma$-formulae} are defined inductively:
\begin{itemize}
    \item Any $\Sigma$-literal is a $\Sigma$-formula.
    \item If $\psi$ is a $\Sigma$-formula so are $\left(\neg\psi\right)$ and $\exists x\ \psi$ with $x\in\mathbf{X}$.
    \item If $\psi$ and $\phi$ are $\Sigma$-formulae so are $\left(\psi\land\phi\right)$ and $\left(\psi\lor\phi\right)$.
\end{itemize}
\end{definition}

Given this syntax the language's semantic can now be defined in a similar manner as for Boolean algebra using interpretations:

\begin{definition}[$\Sigma$-interpretation]
For a signature $\Sigma$ and a set $X\subseteq\mathbf{X}$ of variables with sorts in $\Sigma^S$ a \textit{$\Sigma$-interpretation} over $X$ is a tuple
$\mathcal{I}=\left(\mathcal{U},\mathcal{I}^S, \mathcal{I}^X, \mathcal{I}^F, \mathcal{I}^P\right)$
where:
\begin{itemize}
    \item $\mathcal{U}\neq\emptyset$ is the universe of all possible values;
    \item $\mathcal{I}^S\colon\Sigma^S\to2^\mathcal{U}\footnotemark$ 
        maps each sort $\sigma_i$ to a domain $D_i\coloneqq\mathcal{I}^S\left(\sigma_i\right)$ of possible values for $\Sigma$-terms of this sort;
        \footnotetext{$2^\mathcal{U}$ is the powerset of $\mathcal{U}$}
    \item $\mathcal{I}^X\colon X\to \mathcal{U}$ maps each variable $x\in X$ to a value $v\in\mathcal{U}$;
    \item $\mathcal{I}^F$ maps any function symbol $f\in\Sigma^F$ of rank  $\mathcal{s}^F\left(f\right)=\sigma_1\dotsi\sigma_n\sigma_{n+1}$ to a function $f^\mathcal{I}\colon D_1\times\dotsi\times D_n \to D_{n+1}$; and
    \item $\mathcal{I}^P$ maps any predicate $p\in\Sigma^P$ of rank $\mathcal{s}^P\left(p\right)=\sigma_1\dotsi\sigma_n$ to a truth function $p^\mathcal{I}\colon D_1\times\dotsi\times D_n \to \{0,1\}$.
\end{itemize}
$\mathcal{I}^X$ must respect the sort $\sigma_i$ of $x$ (i.e., $x$ of sort $\sigma_i$ may only be mapped to values $v\in D_i$)
\end{definition}
For terms $t$, variables $v$ or predicates $p$ we denote their value according to a $\Sigma$-interpretation $\mathcal{I}$ as $t^\mathcal{I}$, $v^\mathcal{I}$ and $p^\mathcal{I}$

\begin{definition}[Substitution]
For some interpretation $\mathcal{I}=\left(\mathcal{U},\mathcal{I}^S, \mathcal{I}^X, \mathcal{I}^F, \mathcal{I}^P\right)$ over variables $X\subseteq\mathbf{X}$ with $x\in X$ of type $\sigma_i$ and $a\in D_i$ we define $\mathcal{I}\left[x\mapsto a\right]$ as the interpretation $\left(\mathcal{U},\mathcal{I}^S, \overline{\mathcal{I}^X}, \mathcal{I}^F, \mathcal{I}^P\right)$ with:\\
$\overline{\mathcal{I}^X}(v)=
\begin{cases}
a & v=x\\
\mathcal{I}^X(v) & \textit{otherwise}
\end{cases}$
\end{definition}

\begin{definition}[$\Sigma$-model]
A $\Sigma$-interpretation $\mathcal{I}$ is a \textit{$\Sigma$-model} for some formula $\phi$ if $\mathcal{I}$ \textit{satisfies} $\phi$ (i.e., $\mathcal{I}\vDash\phi$).\\
The \textit{satisfies} relation $\vDash$ can be defined inductively:
\begin{itemize}
    \item $\mathcal{I} \nvDash \square$ and $\mathcal{I} \vDash \blacksquare$.
    \item For $\Sigma$-terms $t_1, t_2$: 
    $\mathcal{I}\vDash t_1\doteq t_2$ iff $t_1^\mathcal{I}=t_2^\mathcal{I}$. 
    \item For $\Sigma$-terms $t_1,\dots,t_n$ and a suitable predicate $p\in\Sigma^P$:\\
    $\mathcal{I}\vDash p\left(t_1,\dots,t_n\right)$ iff $p^\mathcal{I}(t_1^\mathcal{I},\dots,t^\mathcal{I}_n)=1$. 
    \item For some $\Sigma$-formula $\psi$:
    $\mathcal{I}\vDash \neg\psi$ iff $\mathcal{I} \nvDash \psi$.
    \item For $\Sigma$-formulae $\phi, \psi$:
    $\mathcal{I}\vDash \phi\land\psi$ iff $\mathcal{I}\vDash\phi$ and $\mathcal{I}\vDash\psi$.
    \item For $\Sigma$-formulae $\phi, \psi$:
    $\mathcal{I}\vDash \phi\lor\psi$ iff $\mathcal{I}\vDash\phi$ or $\mathcal{I}\vDash\psi$. 
    \item For a $\Sigma$-formula $\psi$:\\
    $\mathcal{I}\vDash \exists x\colon\!\sigma_i\ \psi$ iff $\mathcal{I}\left[x\mapsto a\right] \vDash \psi$ for some $a\in D_i$.
\end{itemize}
\end{definition}

Other logic connectors can be transformed into formulae in the form described above. In particular $\forall x\ \psi$ can be written as $\neg\exists x\ \neg\psi$.\\
Furthermore $\exists x\ \psi$ can be written as $\exists x\colon\!\sigma\ \psi$ in order to state that $x$ is of sort $\sigma$.

\subsection{Theories}
Given the generic many-sorted first order logic defined above we can now define certain theories which constrain the interpretation of various predicates in $\Sigma^P$ or functions in $\Sigma^F$ to values in correspondence with the theory's desired behaviour. We describe 3 theories presented in \cite{Barrett-Tinelli-SMT} and explain how these theories can be intertwined yielding SMT-LIB theories \cite{BarFT-SMTLIB}.

\begin{definition}[$\Sigma$-Theory]
A \textit{$\Sigma$-Theory} is a tuple $T=\left(\Sigma,A\right)$ where $\Sigma$ is a signature and $A$ is a set-theoretical class of interpretations.
\end{definition}

\begin{definition}[T-interpretation, T-satisfiability, T-model, T-entails]
Given a theory $T=\left(\Sigma,A\right)$ any $\Sigma$-interpretation $\mathcal{I}$ is a \textit{T-interpretation} if $\mathcal{I}\in A$.\\
A formula is \textit{T-satisfiable} if it is satisfied by some T-interpretation.\\
A T-interpretation which satisfies some formula $\psi$ is a \textit{T-model} for this formula (i.e., $\mathcal{I}\vDash_T\psi$).\\
A set $\Phi$ of formulae \textit{T-entails} a formula $\psi$ (i.e., $\Phi\vDash_T\psi$) iff every T-interpretation that satisfies $\Phi$ also satisfies $\psi$.
\end{definition}
While omitted in the definitions above for conciseness, it is worth to note that an interpretation $\mathcal{I}$ may be a T-model for some formula $\psi$ even if $\mathcal{I}$ also defines values for functions, predicates or even sorts not in the signature of $\psi$. Among other benefits this can be useful when combining theories.\\
From hereon we denote the set of all $\Sigma$-formulae of some theory $T$ as $\textsc{For}^1_T$.

\paragraph{Bitvectors: QF\_BV}
The quantifier free (QF) theory of fixed size bit vectors (BV), abbreviated as \texttt{QF\_BV}, concerns with the modeling of hardware and low-level software through bitvectors.\\
For every $n\geq1$ \texttt{QF\_BV} contains a sort $\mathbf{BV}_n$ for bitvectors of length $n$. Any bit of such a vector may be either $0$ or $1$.
In terms of function symbols the \texttt{QF\_BV} theory contains extraction (typically written as $x[i]$ to extract a single bit and $x[i\colon j]$ to extract bits $j$ to $i$ with $i>j$) and concatenation (typically written as $x\circ y$) functions as well as a variety of well-known functions implemented in modern hardware and software (e.g. addition, multiplication, division, shifts, negation, and, or, xor etc.). In its general version \texttt{QF\_BV} (encoded in binary) is \textsc{NExpTime}-complete \cite{KovasznaiFroehlichBiere-SMT12}.

\paragraph{Arrays: QF\_ABV}
Adding arrays to the \texttt{QF\_BV} theory results in a theory known as \mbox{\texttt{QF\_ABV}}. The \enquote{stand-alone} extensional array theory contains as sorts $\mathbf{A},\mathbf{I}$ and $\mathbf{E}$ (for \textbf{a}rray, \textbf{i}ndex and array \textbf{e}lements) however $\mathbf{I}$ and $\mathbf{E}$ are bitvectors in \texttt{QF\_ABV}. In terms of function symbols a \texttt{read} as well as a \texttt{write} function are provided with $\mathcal{s}^F(\text{read})=\mathbf{A}\mathbf{I}\mathbf{E}$ and $\mathcal{s}^F(\text{write})=\mathbf{A}\mathbf{I}\mathbf{E}\mathbf{A}$.\\
The array theory is defined through 3 axioms which any signature must satisfy to support the (extensional) array theory:
\begin{equation}
    \tag{A1}
    \label{eq:preliminaries:smt:a1}
    \forall a\colon\!\textbf{A}\ \forall i\colon\!\textbf{I}\ \forall e\colon\!\textbf{E}\ 
    \text{read}\left(\text{write}\left(a,i,e\right), i\right) = e
\end{equation}
\begin{equation}
    \tag{A2}
    \label{eq:preliminaries:smt:a2}
    \forall a\colon\!\textbf{A}\ \forall i\colon\!\textbf{I}\ \forall e\colon\!\textbf{E}\ 
    i \neq j \implies \text{read}\left(\text{write}\left(a,i,e\right), j\right)
    = \text{read}\left(a,j\right)
\end{equation}
\begin{equation}
    \tag{A3}
    \label{eq:preliminaries:smt:a3}
    \forall a\colon\!\textbf{A}\ \forall b\colon\!\textbf{A}\ 
    \left( \forall i\colon\!\textbf{I}\ \text{read}\left(a,i\right) = \text{read}\left(b,i\right) \right)
    \implies
    a=b
\end{equation}
The non-extensional array theory can be obtained by dropping axiom (A3).

\paragraph{Uninterpreted Functions: QF\_AUFBV}
The most simple theory imaginable is a theory containing nothing but arbitrary (uninterpreted) functions with equality. This theory can obviously be used to describe the application of functions on a wide range of objects. However, allowing arbitrary uninterpreted function symbols with bitvectors as parameters and outputs integrates the theory of uninterpreted functions (UF) into \texttt{QF\_ABV} resulting in the \texttt{QF\_AUFBV} theory.\\
This theory can be axiomatized throught the \textit{function congruence axiom} \cite{PreinerNiemetzBiere-DIFTS13}:
\begin{equation}
    \tag{EUF}
    \label{eq:preliminaries:smt:euf}
    \forall \bar{x},\bar{y}\ \bigwedge\limits_{i=1}^n x_i=y_i \implies f\left(\bar{x}\right) = f\left(\bar{y}\right)
\end{equation}


\section{SMT solving}
Over time, a wide range of solvers has evolved making use of a variety of techniques to solve SMT formulae. While we refer the reader to \cite{Barrett-Tinelli-SMT} for a more complete overview of the various techniques used in SMT-solving the following sections will give a brief overview of two predominant methodologies used in SMT-solving - namely Eager and Lazy SMT Solving 
\subsection{Eager SMT solving}
Eager SMT solving is based on the idea of reducing a $\Sigma$-formula $\phi$ of some theory $T$ into a Boolean algebra formula which can be solved by an off-the-shelf SAT solver.\\
This process can be described as a function tuple $(e,t)$ where $e\colon\textsc{For}^1_T\to\textsc{For}^0$ and $t$ maps Boolean interpretations $\mathcal{I}^0$ to $T$-interpretations $\mathcal{I}^1_T$. For the eager SMT solving to work, functions $(e,t)$ are needed such that $\mathcal{I}^0\vDash e\left(\phi\right)$ iff $t\left(\mathcal{I}^0\right)\vDash\phi$.
\par
A good example of the eager SMT solving approach is \textit{bit blasting} which solves bitvector arithmetic problems from theories like \texttt{QF\_BV} by assigning one Boolean variable for each bit appearing in an instance and then implements all functions used in the problem through the logic connectors available in Boolean algebra.
\begin{example}[Addition]
Given two variables $x\colon\mathbf{BV}_2, y\colon\mathbf{BV}_2$ and the formula $11\doteq add\left(x,y\right)$,
the SMT problem can be transformed into Boolean algebra by writing out the addition:
\begin{align*}
1 \leftrightarrow & x[0] \oplus y[0] & \textit{Calculating bit 0}\\
1 \leftrightarrow & x[1] \oplus y[1] \oplus \left(x[0] \land y[0] \right) & \textit{Calculating bit 1}
\end{align*}
This problem can then be solved by an off-the-shelf SAT solver.
\end{example}

\subsection{Lazy SMT Solving}
It is easy to see that the eager approach described above can lead to very large Boolean formula sets. In particular complicated operations like multiplication or division instantly lead to a large number of formulae being added to the SAT solver instance. For this reason there exist a variety of approaches making use of SAT solvers to solve the overall structure (i.e., the abstraction explained below) of the problem while leaving the resolution of theory specific variable assignments to theory specific solvers.

\paragraph{Abstractions}
Given an infinite set $\Pi$ of propositional variables we define a mapping $(\cdot)^a$ which provides an injective mapping of atomic $\Sigma$-formulae to $\Pi$. For quantifier-free $\Sigma$-formulae we can then define $\left(\neg\psi\right)^a=\neg\left(\psi\right)^a$, $\left(\psi\land\phi\right)^a=\psi^a\land\phi^a$ and $\left(\psi\lor\phi\right)^a=\psi^a\lor\phi^a$. Given some SMT formula $\Phi$ this allows us to give the overall structure (i.e., $\Phi^a$) to an off-the-shelf (incremental) SAT solver which will return a model for $\Phi$ containing a set $S=\{\psi_1^a,\dots,\psi_n^a\}$ of atomic $\Sigma$-formulae which need to be satisfied in this specific assignment.

\paragraph{Theory Solver}
A theory specific solver can then check whether the conjunction of $S$ (specifically the conjunction of the original atomic $\Sigma$-formulae) has a satisfiable assignment or not. If it does, $\Phi$ is satisfiable, if not a clause \[\bigvee\limits_{\psi\in S} \neg\psi\] is added to the SAT solver instance and another satisfying assignment of $\Phi^a$ can be searched to run the theory solver on. This can be repeated until the SAT solver either finds an assignment also satisfiable for the theory solver or finds the abstracted instance $\Phi^a$ and its added constraints to be unsatisfiable in which case $\Phi$ is unsatisfiable, too.

\paragraph{Algorithm}
To make the remarks made above clearer Algorithm \ref{algorithm:preliminaries:smtsolving:lazy_algo} gives a glimpse on the inner workings of Lazy SMT solvers where \textit{get\_model} is a call to some (incremental) SAT solver and $\textit{check\_sat}_T$ is a call to the solver for theory $T$.
Note that $A^c$ returns the concrete atoms from their abstracted versions generated through $\psi^a$ as previously explained.

\begin{algorithm}
    \caption{A lazy SMT solving algorithm as presented in \cite{Barrett-Tinelli-SMT}}
    \begin{algorithmic}
    \label{algorithm:preliminaries:smtsolving:lazy_algo}
    \REQUIRE $\psi$ is a quantifier free $\Sigma$-formula of theory $T$
    \ENSURE output is \texttt{sat} if $\psi$ is $T$-satisfiable, and \texttt{unsat} otherwise
    \STATE $F \coloneqq \psi^a$
    \LOOP
        \STATE $A \coloneqq \textit{get\_model}\left(F\right)$
        \IF{$A = \texttt{none}$}
            \RETURN \texttt{unsat}
        \ELSE
            \STATE $\mu \coloneqq \textit{check\_sat}_T\left(A^c\right)$
            \IF{$\mu = \texttt{sat}$}
                \RETURN \texttt{sat}
            \ELSE
                \STATE $F \coloneqq F \land \neg \mu^a$
            \ENDIF
        \ENDIF
    \ENDLOOP
    \end{algorithmic}
\end{algorithm}