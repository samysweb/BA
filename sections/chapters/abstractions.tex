\chapter{Refinement approach}
\label{ch:refinement_approach}
In this work we present a per-operation refinement approach for the quantifier free bitvector theory (QF\_BV) which substitutes specific operations (e.g. \texttt{bvmul}, \texttt{bvsdiv} or \texttt{bvsrem}) by abstracted versions of them in QF\_UFBV theory and then refines its constraints iteratively until the underlying SMT-Solver either returns unsatisfiability or satisfiability with sound variable assignments. We thereby hope to speedup the decision procedure for problems which are currently \enquote{hard} to solve.

\section{Abstraction Scheme}
\label{sec:refinement_approach:abstraction_scheme}
\todo{Recheck in what way this is the same as CEGAR}
For each SMT-LIB instruction \texttt{op} (e.g. \texttt{bvmul}) that we want to abstract we define uninterpreted functions $ab_{\texttt{op}}^w\left(\dots\right)$ for each $w\in W$ where $W$ are the bitwidth sorts in which \texttt{op} appears.
Given an SMT-LIB input formula $\phi$ we can then substitute each invocation of the form \texttt{(op arg\_1 \dots arg\_n)} by the invocation of the uninterpreted function $ab_{\texttt{op}}^w\left(\texttt{arg\_1}, \dots, \texttt{arg\_n}\right)$.

We define $\hat{\phi}$ as the formula $\phi$ in which all appearances of relevant operations \texttt{op} have been replaced by their abstractions $ab_\texttt{op}^w$.\\

\begin{definition}{Abstraction}

For each instruction \texttt{op} and for each bitwidth $w\in W$  we further define a set of formulas
\[\mathcal{A}_\texttt{op}^w \subset \{f \in \mathcal{F}_{QF\_UFBV} \mid \left( ab_{\texttt{op}}^w\left(\texttt{arg\_1}, \dots, \texttt{arg\_n}\right)=\texttt{(op arg\_1 \dots arg\_n)} \right) \Rightarrow f\}\]
with a total order
\[\mathcal{A}_\texttt{op}^w  = \{a_1, \dots, a_m\}\]
such that
\[\bigwedge\limits_{a\in\mathcal{A}_\texttt{op}^w} a \Rightarrow \left(ab_{\texttt{op}}^w\left(\texttt{arg\_1}, \dots, \texttt{arg\_n}\right)=\texttt{(op arg\_1 \dots arg\_n)}\right)\]
We call the set $\mathcal{A}_\texttt{op}^w$ the abstraction of \texttt{op} with bitwidth $w$.
\end{definition}

\begin{lemma} Equivalence of abstraction
\label{lemma:refinement_approach:abstraction_scheme:equivalence}
\begin{enumerate}
    \item \label{lemma:refinement_approach:abstraction_scheme:equivalence:1} Given a formula $\phi$ with exactly one appearance of the instruction \texttt{op} with bitwidth $w$ as defined above:
    \[
        \bigwedge\limits_{a\in\mathcal{A}_\texttt{op}^w} a \land \hat{\phi}
    \]
    has a model iff $\phi$ has a model
    \item For multiple appearances of \texttt{op} or appearances of multiple different instructions \texttt{op}, \texttt{op'}, \dots we can show the same result iteratively by substituting one instruction after another
\end{enumerate}
\begin{proof}
We only proof 1 as 2 is a trivial result of 1
\begin{itemize}
    \item[$\Leftarrow$]
    We assume there is a model $\mathcal{M}$ for which $\phi$ is true. In this case we can extend the model of $\phi$ by defining a model $\mathcal{M}'$ with $ab_\texttt{op}^w$ acting just like the operation \texttt{op}. In this case $\hat{\phi}$ is obviously satisfiable with the model $\mathcal{M}'$ and all $a \in \mathcal{A}_\texttt{op}^w$ are satisfiable by definition as $\left( ab_{\texttt{op}}^w\left(\texttt{arg\_1}, \dots, \texttt{arg\_n}\right)=\texttt{(op arg\_1 \dots arg\_n)} \right)$ is true in $\mathcal{M}'$.
    \item[$\Rightarrow$]
    We assume a model $\mathcal{M}$ for which $\hat{\phi}$ and all $a \in \mathcal{A}_\texttt{op}^w$ are true.
    By definition of $\mathcal{A}_\texttt{op}^w$ we know that $\left( ab_{\texttt{op}}^w\left(\texttt{arg\_1}, \dots, \texttt{arg\_n}\right)=\texttt{(op arg\_1 \dots arg\_n)} \right)$ is true as well for model $\mathcal{M}$.
    This again implies that $\mathcal{M}$ is also a model for $\phi$ as the results of $ab_\texttt{op}^w$ and \texttt{op} are indistinguishable in its results.
\end{itemize}
\end{proof}
\end{lemma}

With those basic definitions at hand one can construct simple (Algorithm \ref{algorithm:refinement_approach:abstraction_scheme:refinement}) or arbitrarily complicated, heuristic  refinement decision procedures which make use of the above mentioned abstractions. In the following sections we will look at abstractions of various SMT-LIB instructions which will then be evaluated in the next chapter.

\begin{algorithm}
    \caption{Decision procedure for QF\_BV abstractions}
    \begin{algorithmic}
    \label{algorithm:refinement_approach:abstraction_scheme:refinement}
    \REQUIRE $\phi \in \mathcal{F}_{QF\_BV}$
    \STATE abstractions$ \leftarrow \langle\rangle$
    \STATE operations$ \leftarrow \langle\rangle$
    \FORALL{\texttt{op} in $\phi$}
        \IF{\texttt{op} needs refinement}
            \STATE Replace \texttt{op} by $ab_\texttt{op}^w$
            \STATE abstractions.push$\left(\mathcal{A}_\texttt{op}^w\right)$
            \STATE operations.push$\left(\texttt{op}\right)$
        \ENDIF
    \ENDFOR
    \STATE \texttt{ADD\_CLAUSES($\phi$)}
    \LOOP
    \STATE $r \leftarrow $\texttt{SAT($\phi$)}
    \IF{\NOT $r$}
        \PRINT unsat
    \ELSE
        \STATE sound $\leftarrow \TRUE$
        \FORALL{\texttt{op} in operations}
            \IF{\NOT \texttt{op} assignment is sound}
                \STATE sound $\leftarrow \FALSE$
            \ENDIF
        \ENDFOR
        \IF{sound}
            \PRINT sat
        \ELSE
            \FORALL{A in $l1$}
                \STATE \texttt{ADD\_CLAUSES(A.pop())}
            \ENDFOR
        \ENDIF
    \ENDIF
    \ENDLOOP
    \end{algorithmic}
\end{algorithm}

\section{Refining \texttt{mul}}
Generally speaking we considered three different approaches for the abstraction of the \texttt{mul} instruction which work on different levels.
For two of the proposed abstractions we will need a way of estimating whether
\subsection{Simple cases}

\subsection{Most significant digit based intervals}

\subsection{Relations to other instructions}

\section{Refining \texttt{sdiv}}

\section{Refining \texttt{srem}}

