\chapter{Implementation}
\label{ch:implementation}

Generally speaking there were 2 pathways one could have taken for the experiments:
Either implement the abstraction refinement procedures directly into an existing SMT solver or build a layer on top of some SMT solver which then implements the abstraction refinement procedure.
The big advantage of an implementation directly within an SMT solver being the performance enhancements through optimizations and integrations only possible from within
(e.g. working directly inside the over- and under-approximation loop of Boolector described in \ref{par:related_work:boolector}).
The disadvantage however would have been a much slower evaluation process due to the enhanced complexity and steeper learning curve when implementing new features
in a project similarly scaled to Boolector.
\par
As the focus of this work lay on researching which abstractions are effective in reducing the solvers runtime and not so much on building an improved solver,
the abstraction refinement procedure was built as a layer on top of Boolector.
In order enable a fast evaluation of the abstraction approaches, the prototype was implemented in Python.
To increase readibility we will sometimes refer to the prototype as \enquote{Ablector} which stands for \textit{Ab}stracted Boo\textit{lector}

\section{Architecture}
While Boolector offers a Python Interface (pyboolector) on its own, this Python interface is difficult to enhance when it comes to parsing SMT-LIB files
as pyboolector moves directly into C-level parsing when evaluating SMT-LIB files and only returns to the Python level when the solver's result can be returned.
Therefore another SMT-LIB parser was necessary to allow for the abstraction refinement to be implemented in Python.
This resulted in the architecture in figure \ref{fig:implementation:architecture} which will be explained below.

\begin{figure}[h!]
    \centering
    \begin{tikzpicture}[node distance=2cm]
    \tikzstyle{io} = [minimum width=3cm, minimum height=1cm, text centered]
    \tikzstyle{process} = [rectangle, minimum width=3cm, minimum height=1cm, text centered, draw=black]
    \tikzstyle{decision} = [diamond, minimum width=3cm, minimum height=1cm, text centered, draw=black]
    \tikzstyle{arrow} = [thick,->,>=stealth]
    
    \node (in) [io] {Formula};
    \node (out) [io, right of=in, xshift=1.5cm] {SAT/UNSAT};
    \node (pysmt) [process,below of=in,align=center] {PySMT:\\SMT-LIB parsing};
    \node (ablector) [process, below=of pysmt.west,anchor=west,yshift=-0.5cm] {Ablector: Abstraction refinement};
    \node (boolector) [process, below of=ablector,yshift=-0.3cm,fill=white] {Boolector: Incremental SMT solving};
    
    \node (refine) [process, fit=(ablector) (boolector),inner sep=0.8cm] {};
    \node[fill=white] at (refine.north) {Refinement loop};
    \begin{scope}[on background layer]
        \draw[dashed] ([xshift=-4cm]boolector.west) -- node [anchor=north west, align=left, pos=0] {C Level} node [anchor=south west, align=left, pos=0] {Python Level} ([xshift=4cm]boolector.east);
    \end{scope}
    \draw [arrow] (in) -- (pysmt);
    \draw [arrow] ([xshift=-1cm]pysmt.south) -- node[anchor=east,align=right,fill=white,xshift=-0.1cm] {SMT instructions} ([xshift=-1cm]pysmt.south |- ablector.north) ;

    \draw [arrow] ([xshift=-2cm]ablector.south) -- node[anchor=east,align=right, fill=white,xshift=-0.2cm] {abstracted\\formulae} ([xshift=-2cm]boolector.north);
    \draw [arrow] ([xshift=+2cm]boolector.north) -- node[anchor=west, fill=white,xshift=0.2cm] {SAT/UNSAT} ([xshift=+2cm]ablector.south);
    
    \draw [arrow] ([xshift=+2cm]ablector.north) -- ([xshift=+2cm]ablector.north |- out.south);
    
    \end{tikzpicture}
    \caption{The overall architecture of the prototype implemented}
    \label{fig:implementation:architecture}
\end{figure}


\subsection{PySMT}
\label{sec:implementation:pysmt}
As parser the PySMT Python library \cite{pysmt2015} was used. The library allows
SMT-LIB parsing and integrates interfaces for various SMT-solvers including Boolector.
The library made it possible to enhance the Boolector interface in a way that allowed the implementation
of the abstraction refinement procedure.\\
The downside of this library was its failing to parse a few benchmarks containing certain bitshift operations
and the relatively slow parsing time in comparison with Boolector's parser\footnotemark.
\footnotetext{Please note that this result was founding using some very basic tracing techniques and therefore might not give a full picture of PySMT's parsing performance.
In any case the parsing times of Boolector and PySMT seemed to differ sufficiently for this to potentially have an impact on the performance comparisons later on.}

\subsection{Boolector}
\label{sec:implementation:boolector}
As underlying SMT solver Boolector \cite{Brummayer-Biere2009_Chapter_BoolectorAnEfficientSMTSolverF} was used
which won both the \texttt{QF\_ABV} and \texttt{QF\_BV} main track at SMT-COMP 2018 \cite{SMTCOMP18}.
The solver integrates with PySMT through its Python interface pyboolector and supports uninterpreted functions.
For the implementation of the refinement loop the solver's python interface was extended through a subclass rewriting certain methods of pyboolector
(like the method for multiplication and the method for signed division and remainder) in order to implement the abstraction procedures.
As PySMT integrates the invocation of pyboolector's methods into its own parsing procedure this methodology resulted in a setup allowing to
easily test out abstractions.
\par
As already mentioned earlier Boolector makes heavy use of various optimization techniques before bit blasting.
Oftentimes Boolector can actually solve SMT instances entirely without bit blasting.
As these optimization techniques are rather quick in runtime we actually run them on every instance before the abstraction refinement procedure to check if these
procedures already find the instance to be unsatifiable.
Only if the instance is not yet found to be unsatisfiable by these procedures we continue and run our abstraction procedure on the original (that is, not the optimized) version of the SMT instance.

\subsection{Abstraction node managment}
Upon invocation of a method like \texttt{Mul$\left(\cdot,\cdot\right)$} the pyboolector interface returns an object representing the result of the
multiplication function which can again be passed into other pyboolector methods.
The extending subclass keeps a list of so called \textit{abstraction nodes}. Upon invocation of a rewritten method like \texttt{Mul$\left(\cdot,\cdot\right)$} a new \textit{abstraction node}
is added to this list. The \textit{abstraction node} of some function (e.g. \texttt{Mul$\left(\cdot,\cdot\right)$}) contains the entire logic necessary in the abstraction refinement steps.
Among other methods any \textit{abstraction node} must therefore contain the following methods:
\begin{itemize}
    \item \texttt{isExact$\left(\right)$}\\
    Returns whether the given node is already sound
    \item \texttt{isCorrect$\left(\right)$}\\
    Returns whether the current assignment of the given node is correct (i.e., a correct result of the function given the current input assignments).
    \item \texttt{refine$\left(\right)$}\\
    Prepares the assertions necessary for the next refinement step
    \item \texttt{doAssert$\left(\right)$}\\
    Adds the assertions prepared in \texttt{refine$\left(\right)$} to the solver instance.
\end{itemize}
This allows an easy abstraction of any function and makes the abstraction managment of the various functions independent from each other.
The refinement loop can then call these methods on each \textit{abstraction node} as needed.\\
Once an abstraction node returns that it is exact (i.e., both complete and sound), it is removed from the list of \textit{abstraction nodes} and no longer visited in the refinement steps.

\section{Time measurements}
Like we already mentioned in the PySMT section above the parsing engine of PySMT is notably slower than the parsing engine of Boolector.
Comparing the real time (i.e., clock time) of the two runs would therefore result in a very biased performance comparison as we are trying
to benchmark the abstractions and not the parsing.\\
We therefore decided to only compare the CPU clock time of certain operations.
Specifically we compare the CPU clock time\footnote{We chose the CPU clock time as it can be measured through the same unified interface in both Python and C with comparable results. As the tasks in this program section are heavily CPU-bound with very little IO operations taking place, this measurement can still be considered realistic.} of Boolector's SMT-LIB \texttt{check-sat} call
against the summed up CPU clock time of all invocations to rewritten procedures in Ablector including its own \texttt{check-sat} call.
This will hopefully produce a more realistic comparison of the abstraction's performance.
Note that we are over-approximating the time Ablector takes in this comparison as we are
also adding the processing time for abstracted functions (e.g. \texttt{Mul$\left(\cdot,\cdot\right)$}) which are not considered for Boolector.
However, those time measurements are in most cases negligible in comparision to the time for the \texttt{check-sat} call.\\
This time measurement will usually be refered to as \texttt{satpart} as it is essentially measuring the time the solver takes to produce an (un)sat result without parsing.