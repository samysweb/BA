\chapter{Conclusion}
\label{ch:Conclusion}
In this work, we introduced a novel approach to solving quantifier free bitvector problems in SMT-LIB's \texttt{QF\_BV} theory. The approach is based on abstraction methodologies previously used for various other problems in logic and specifically in SMT.
On the one hand, we presented numerous abstractions for 3 comparatively costly funcitons of the bitvector theory, on the other hand, we proposed a simple theoretical framework allowing a proof of correctness for the presented abstractions.
While we saw that the presented approach performs better than Boolector in deciding unsatisfiable bitvector problems, solving 43 instances more, the implemented prototype is not yet competitive for satisfiable instances. This is of course in some way a natural result, as over-approximations usually improve the solver runtime on unsatisfiable (and not on satisfiable) instances.

\paragraph{Future Work}
With the abstraction's correctness proved and the abstraction's performance evaluated through the current prototype, one could now implement the abstraction scheme directly into a SMT solver like Boolector making use of interleaved under- and over-approximations. We expect that this might enhance the solver's performance sufficiently to be competitive for both satisfiable and unsatisfiable instances.
Additionally, a time limit for each refinement round could be introduced in order to avoid cases where the solver gets stuck in some specific refinement step. For this, a detailed parameter analysis will be necessary to find a time limit that keeps the abstraction steps effective while avoiding dead ends. Finally, it is left to investigate whether a \textit{don't care reasoning} strategy similar to the one used for Lemmas on Demand in Boolector \cite{NiemetzPreinerBiere-FMCAD14} could improve the abstraction refinement procedure's performance.